% Working Paper Template
% This template provides a starting point for mathematical research working papers
% Customize as needed for your specific project

\documentclass{article}

% Standard mathematical packages
\usepackage{amsmath}
\usepackage{amsthm}
\usepackage{amssymb}

% Smart cross-referencing
\usepackage{cleveref}

% Bibliography management
\usepackage[backend=biber,style=alphabetic]{biblatex}
\addbibresource{references.bib}

% Theorem environments
\newtheorem{theorem}{Theorem}
\newtheorem{lemma}[theorem]{Lemma}
\newtheorem{proposition}[theorem]{Proposition}
\newtheorem{corollary}[theorem]{Corollary}

\theoremstyle{definition}
\newtheorem{definition}[theorem]{Definition}
\newtheorem{example}[theorem]{Example}

\theoremstyle{remark}
\newtheorem{remark}[theorem]{Remark}
\newtheorem{note}[theorem]{Note}

% Cleveref names for custom environments (if you add any)
% Standard environments (theorem, lemma, etc.) are handled automatically
% Example for assumption environment:
% \newtheorem{assumption}[theorem]{Assumption}
% \crefname{assumption}{Assumption}{Assumptions}

% ============================================================================
% SEMANTIC MACROS
% ============================================================================
% Define semantic macros here to make variables meaningful and easy to refactor
% Naming convention: Use compact, prefix-free names
% Examples:
%   - Measures: \measa, \measb, \measmu, \measnu
%   - Vectors: \va, \vb, \vx, \vy
%   - Sets: \seta, \setb, \setx
%   - Functions: \fa, \fb, \fmap
%   - Operators: \opa, \opb, \opt
%
% Keep macros compact (token-efficient) and prefix-free (enables clean search-and-replace)
% ============================================================================

% Example measure macros
\newcommand{\measa}{\mu}
\newcommand{\measb}{\nu}

% Example vector macros
\newcommand{\va}{\mathbf{a}}
\newcommand{\vb}{\mathbf{b}}

% Example set macros
\newcommand{\seta}{\mathcal{A}}
\newcommand{\setb}{\mathcal{B}}

% Add your semantic macros below:
% \newcommand{\yourmacro}{...}


\title{Working Paper Title}
\author{Author Name}
\date{\today}

\begin{document}

\maketitle

\begin{abstract}
Brief description of the main results and approach.
\end{abstract}

% ============================================================================
\section{Introduction}
% ============================================================================

Motivate the problem and outline the approach.


% ============================================================================
\section{Definitions and Preliminaries}
% ============================================================================

\begin{definition}
\label{defn:example}
Define key concepts here.
\end{definition}


% ============================================================================
\section{Main Results}
% ============================================================================

\begin{theorem}
\label{thm:main-result}
State main theorem here using the notation from \cref{defn:example}.
\end{theorem}

\begin{proof}
Proof goes here. For incomplete proofs, use proof sketches with explicit TODOs.
\end{proof}


% ============================================================================
\section{Open Questions and Failed Approaches}
% ============================================================================
% Document approaches that didn't work and why - this helps avoid revisiting
% dead ends and may provide insights for readers attempting similar problems.

\subsection{Failed Approach: [Description]}

% Describe the approach that was attempted and why it failed


% ============================================================================
% Bibliography
% ============================================================================

\printbibliography

\end{document}
